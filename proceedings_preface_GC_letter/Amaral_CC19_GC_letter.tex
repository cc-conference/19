\documentclass[12pt,letterpaper]{article}

% encoding and fonts first
\usepackage[utf8]{inputenc}
\usepackage[T1]{fontenc}
\usepackage{microtype}
\usepackage[tt=false, type1=true]{libertine}
\sloppy

\usepackage{geometry}
%% For preface:
\geometry{textwidth=14cm,textheight=20cm}
%% For committee listings and sponsor pages:
%\geometry{textwidth=18cm,textheight=23.5cm}

%% For format `acmsmall'
%\geometry{twoside=true,
%          includeheadfoot, head=13pt, foot=2pc,
%          paperwidth=6.75in, paperheight=10in,
%          top=58pt, bottom=44pt, inner=46pt, outer=46pt,
%          marginparwidth=2pc,heightrounded
%         }

%% For IEEE conferences and workshops, please uncomment the following line to use Times.
%\usepackage{times}

%-------------------------------------------------------------------------
\begin{document}

\title{\sffamily\bfseries Welcome from the General Chair}
\date{}

\maketitle
\thispagestyle{empty}
\pagestyle{empty}

As I welcome you to the 28th edition of the International Conference on Compiler Construction in Washington DC --- the fourth edition of CC that is co-located with CGO, PPoPP and HPCA --- I reflect on the significant role that CC has played on my own research. There are many memorable moments. For instance, at my first CC in Genova in April 2001 I discovered that our integrated approach to register allocation and instruction scheduling had been simultaneously and independently developed by Sid Touati Ali at INRIA in France. Then, at CC in Warsaw in 2003 I finally met Sid and, while watching a related presentation at the conference, we started a discussion that led to our formalization of the offset assignment problem and shed new light in the previous research in that area.  

Over the years, there have been many insights, connections, relations that started with interactions at CC. Sometimes inspired by an insightful observation in a presentation, often from a discussion with a colleague during a coffee break or over a meal. I hope that for all the participants for whom this is their first CC that it will be inspirational and rewarding, and I hope to see you again in future editions of CC. For our returning colleagues, welcome back. 

Moving CC to collocate with CGO and associated conferences has aligned CC with complementary conferences and facilitated travel plans for many compiler researchers. Organizationally there are still challenges to be addressed and I have worked with the CC steering committee to try to improve the planing and organization process for future editions. Thus, while we expect that you will have an enjoyable and rewarding event in Washington DC this year, I also expect that the preparation work for the conference shall be streamlined in coming years. Thus, this is a good time to volunteer and get involved in the organization of future versions of CC.

I hope you enjoy your time in Washington and have many fruitful interactions with collegues.

Welcome to CC 2019.

\bigskip
\noindent
Washington DC \hfill J. Nelson Amaral\\
February 2019 \hfill CC 2019 General Chair

\end{document}
