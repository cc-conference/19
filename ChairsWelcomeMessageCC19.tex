\documentclass[12pt,letterpaper]{article}

% encoding and fonts first
\usepackage[utf8]{inputenc}
\usepackage[T1]{fontenc}
\usepackage{microtype}
\usepackage[tt=false, type1=true]{libertine}
\sloppy

\usepackage{geometry}
%% For preface:
\geometry{textwidth=14cm,textheight=20cm}
%% For committee listings and sponsor pages:
%\geometry{textwidth=18cm,textheight=23.5cm}

%% For format `acmsmall'
%\geometry{twoside=true,
%          includeheadfoot, head=13pt, foot=2pc,
%          paperwidth=6.75in, paperheight=10in,
%          top=58pt, bottom=44pt, inner=46pt, outer=46pt,
%          marginparwidth=2pc,heightrounded
%         }

%% For IEEE conferences and workshops, please uncomment the following line to use Times.
%\usepackage{times}


%-------------------------------------------------------------------------
\begin{document}

\title{\sffamily\bfseries Welcome from the Chairs}
\date{}

\maketitle
\thispagestyle{empty}
\pagestyle{empty}

Welcome to the 28th International Conference on Compiler Construction (CC), held in Washington, DC, February 16--17 2019. CC continues its focus on processing programs in the most general sense: analyzing, transforming or executing input that describes how a system operates, including traditional compiler construction as a special case.

This is  the fourth edition of CC that is co-located with CGO, PPoPP and HPCA. 
We hope that for all the participants for whom this is their first CC that it will be inspirational and rewarding, and we hope to see you again in future editions of CC. For our returning colleagues, welcome back. As we thank the organizers of the co-located conferences, the program committee, all the authors and presenters, our sponsors, we also invite you to engage further with the CC community and to volunteer and get involved in the organization of future versions of CC.

We thank Eddie Kohler for his assistance in making sure the process of gathering submissions through HotCRP went smoothly.  Each paper was reviewed by at least three PC members. The review process was double-blind from start to finish: accepted papers were unblinded only once the list of accepted papers was posted, and rejected papers were not unblinded. Overall, we received 45 papers, of which 17 were accepted, for an acceptance rate of 38\%. The PC chair oversaw a subcommittee of PC members to select a best paper for the conference.

The papers are organized into five sessions across one and a half days. We thank all of the volunteers who are chairing these sessions. We also would like to thank Saman Amarasinghe for graciously agreeing to present the CC Keynote on the Sparse Tensor Algebra Compiler.

All material should fit within a rectangle of 18 × 23.5 cm (7" × 9.25"), centred on the page. Short welcome messages can use a smaller text area.
%
%\textbf{ACM}: The titles (Biolinum, sans-serif, 18 point, bold), the headings (Biolinum, sans-serif, 14 point, bold), 
%and the text (Libertine, serif, 12 point) fill the full width of the page – one column.
%\textbf{IEEE}: The titles (Helvetica/Arial, 18 point, bold), the headings (Helvetica/Arial, 14 point, bold), 
%and the text (Times New Roman, 12 point) fill the full width of the page – one column.
%
%No classifiers and no copyright block at the bottom of the first page are needed. No transfer of copyright or permission release is necessary.
%
%Page numbers should be omitted.
%
%Text and URLs should be in black, not in blue/other color, and not underlined.
%
%If the submission system asks for an abstract, please copy the first paragraph of the welcome message (or whatever is appropriate) as abstract. This will be shown as description in HTML navigation structures.
%
%The deadline for submission of this preface is one week after the camera-ready paper deadline for the papers.
%

\bigskip
\noindent
Edmonton, Alberta       \hfill J. Nelson Amaral\\
February 2019 \hfill General Chair
\\
~\\
West Lafayette, IN       \hfill Milind Kulkarni\\
February 2019 \hfill PC Chair

\end{document}
